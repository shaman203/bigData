%%%%%%%%%%%%%%%%%%%%%%%%%%%%%%%%%%%%%%%%%
% Beamer Presentation
% LaTeX Template
% Version 1.0 (10/11/12)
%
% This template has been downloaded from:
% http://www.LaTeXTemplates.com
%
% License:
% CC BY-NC-SA 3.0 (http://creativecommons.org/licenses/by-nc-sa/3.0/)
%
%%%%%%%%%%%%%%%%%%%%%%%%%%%%%%%%%%%%%%%%%

%----------------------------------------------------------------------------------------
%	PACKAGES AND THEMES
%----------------------------------------------------------------------------------------

\documentclass{beamer}

\mode<presentation> {

% The Beamer class comes with a number of default slide themes
% which change the colors and layouts of slides. Below this is a list
% of all the themes, uncomment each in turn to see what they look like.

%\usetheme{default}
%\usetheme{AnnArbor}
%\usetheme{Antibes}
%\usetheme{Bergen}
%\usetheme{Berkeley}
\usetheme{Berlin}
%\usetheme{Boadilla}
%\usetheme{CambridgeUS}
%\usetheme{Copenhagen}
%\usetheme{Darmstadt}
%\usetheme{Dresden}
%\usetheme{Frankfurt}
%\usetheme{Goettingen}
%\usetheme{Hannover}
%\usetheme{Ilmenau}
%\usetheme{JuanLesPins}
%\usetheme{Luebeck}
%\usetheme{Madrid}
%\usetheme{Malmoe}
%\usetheme{Marburg}
%\usetheme{Montpellier}
%\usetheme{PaloAlto}
%\usetheme{Pittsburgh}
%\usetheme{Rochester}
%\usetheme{Singapore}
%\usetheme{Szeged}
%\usetheme{Warsaw}

% As well as themes, the Beamer class has a number of color themes
% for any slide theme. Uncomment each of these in turn to see how it
% changes the colors of your current slide theme.

%\usecolortheme{albatross}
%\usecolortheme{beaver}
%\usecolortheme{beetle}
%\usecolortheme{crane}
%\usecolortheme{dolphin}
%\usecolortheme{dove}
%\usecolortheme{fly}
%\usecolortheme{lily}
%\usecolortheme{orchid}
%\usecolortheme{rose}
%\usecolortheme{seagull}
%\usecolortheme{seahorse}
%\usecolortheme{whale}
%\usecolortheme{wolverine}

%\setbeamertemplate{footline} % To remove the footer line in all slides uncomment this line
%\setbeamertemplate{footline}[page number] % To replace the footer line in all slides with a simple slide count uncomment this line
\setbeamertemplate{headline}{}
\setbeamertemplate{navigation symbols}{} % To remove the navigation symbols from the bottom of all slides uncomment this line
}

  \setbeamertemplate{footline}%{miniframes theme}
  {%
    \begin{beamercolorbox}[colsep=1.5pt]{upper separation line foot}
    \end{beamercolorbox}
    \begin{beamercolorbox}[ht=2.5ex,dp=1.125ex,%
      leftskip=.3cm,rightskip=.3cm plus1fil]{author in head/foot}%
      \leavevmode{\usebeamerfont{author in head/foot}\insertshortauthor}%
      \hfill%
      {\usebeamerfont{institute in head/foot}\usebeamercolor[fg]{institute in head/foot}\insertshortinstitute}%
    \end{beamercolorbox}%
    \begin{beamercolorbox}[ht=2.5ex,dp=1.125ex,%
      leftskip=.3cm,rightskip=.3cm plus1fil]{title in head/foot}%
      {\usebeamerfont{title in head/foot}\insertshorttitle} \hfill     \insertframenumber/\inserttotalframenumber%
    \end{beamercolorbox}%
    \begin{beamercolorbox}[colsep=1.5pt]{lower separation line foot}
    \end{beamercolorbox}
  }



\usepackage{graphicx} % Allows including images
\usepackage{booktabs} % Allows the use of \toprule, \midrule and \bottomrule in tables
\usepackage[utf8]{inputenc}
\usepackage[magyar]{babel}
\usepackage{fixltx2e}
\usepackage{epstopdf}
\usepackage{xcolor}



%----------------------------------------------------------------------------------------
%	TITLE PAGE
%----------------------------------------------------------------------------------------

\title[Big Data elemzési eszközök nyílt forráskódú platformokon]{Big Data elemzési eszközök nyílt forráskódú platformokon\\ Házi feladat} % The short title appears at the bottom of every slide, the full title is only on the title page

\author{Mátyás-Barta Csongor} % Your name

\institute[BME-MIT] % Your institution as it will appear on the bottom of every slide, may be shorthand to save space
{
VYW0YR\\
mbcsongor@yahoo.com \\% Your email address
}
\date{\today} % Date, can be changed to a custom date

\begin{document}


\begin{frame}[plain]
\titlepage % Print the title page as the first slide
\end{frame}

\begin{frame}
\frametitle{Választott feladatok és technológiák} 
\begin{itemize}
	\item Flight1 - Melyik reptéren gurulnak (TaxiIn és TaxiOut) átlagosan legtöbbet a gépek? \\
		Java MapReduce
	\item Flight4 - Melyik 10 légitársaság indul a legtöbbször későn (DepDelay)? \\
		Spark
	\item Flight2 - Honnan szállt fel a legtöbb repülő? \\
		Pig
\end{itemize}

%----------------------------------------------------------------------------------------
%	PRESENTATION SLIDES
%----------------------------------------------------------------------------------------

%------------------------------------------------
\section{Flight1 - Java MapReduce} % Sections can be created in order to organize your presentation into discrete blocks, all sections and subsections are automatically printed in the table of contents as an overview of the talk
%------------------------------------------------

%------------------------------------------------
\begin{frame}
\frametitle{Lépések}
\begin{columns}[t] 
	\column{.45\textwidth} % Left column and width		
		
		
	\column{.5\textwidth} % Right column and width
		
\end{columns}
\end{frame}

%------------------------------------------------

\begin{frame}
\frametitle{Futtatás}

\end{frame}


%------------------------------------------------
\section{Flight4 - Spark} % Sections can be created in order to organize your presentation into discrete blocks, all sections and subsections are automatically printed in the table of contents as an overview of the talk
%------------------------------------------------

%------------------------------------------------
\begin{frame}
\frametitle{Lépések}
\begin{columns}[t] 
	\column{.45\textwidth} % Left column and width		
		
		
	\column{.5\textwidth} % Right column and width
		
\end{columns}
\end{frame}

%------------------------------------------------

\begin{frame}
\frametitle{Futtatás}

\end{frame}


%------------------------------------------------
\section{Flight2 - Pig} % Sections can be created in order to organize your presentation into discrete blocks, all sections and subsections are automatically printed in the table of contents as an overview of the talk
%------------------------------------------------

%------------------------------------------------
\begin{frame}
\frametitle{Lépések}
\begin{columns}[t] 
	\column{.45\textwidth} % Left column and width		
		
		
	\column{.5\textwidth} % Right column and width
		
\end{columns}
\end{frame}

%------------------------------------------------

\begin{frame}
\frametitle{Futtatás}

\end{frame}

%------------------------------------------------

%------------------------------------------------

\begin{frame}[plain]
\Huge{\centerline{Köszönöm a figyelmet!}}
\end{frame}


%------------------------------------------------
\end{document} 
